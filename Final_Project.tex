\documentclass[]{article}
\usepackage{lmodern}
\usepackage{amssymb,amsmath}
\usepackage{ifxetex,ifluatex}
\usepackage{fixltx2e} % provides \textsubscript
\ifnum 0\ifxetex 1\fi\ifluatex 1\fi=0 % if pdftex
  \usepackage[T1]{fontenc}
  \usepackage[utf8]{inputenc}
\else % if luatex or xelatex
  \ifxetex
    \usepackage{mathspec}
  \else
    \usepackage{fontspec}
  \fi
  \defaultfontfeatures{Ligatures=TeX,Scale=MatchLowercase}
\fi
% use upquote if available, for straight quotes in verbatim environments
\IfFileExists{upquote.sty}{\usepackage{upquote}}{}
% use microtype if available
\IfFileExists{microtype.sty}{%
\usepackage{microtype}
\UseMicrotypeSet[protrusion]{basicmath} % disable protrusion for tt fonts
}{}
\usepackage[margin=1in]{geometry}
\usepackage{hyperref}
\hypersetup{unicode=true,
            pdftitle={Predictions using the Weight Lifting Exercises Dataset},
            pdfauthor={Tom Lous},
            pdfborder={0 0 0},
            breaklinks=true}
\urlstyle{same}  % don't use monospace font for urls
\usepackage{color}
\usepackage{fancyvrb}
\newcommand{\VerbBar}{|}
\newcommand{\VERB}{\Verb[commandchars=\\\{\}]}
\DefineVerbatimEnvironment{Highlighting}{Verbatim}{commandchars=\\\{\}}
% Add ',fontsize=\small' for more characters per line
\usepackage{framed}
\definecolor{shadecolor}{RGB}{248,248,248}
\newenvironment{Shaded}{\begin{snugshade}}{\end{snugshade}}
\newcommand{\AlertTok}[1]{\textcolor[rgb]{0.94,0.16,0.16}{#1}}
\newcommand{\AnnotationTok}[1]{\textcolor[rgb]{0.56,0.35,0.01}{\textbf{\textit{#1}}}}
\newcommand{\AttributeTok}[1]{\textcolor[rgb]{0.77,0.63,0.00}{#1}}
\newcommand{\BaseNTok}[1]{\textcolor[rgb]{0.00,0.00,0.81}{#1}}
\newcommand{\BuiltInTok}[1]{#1}
\newcommand{\CharTok}[1]{\textcolor[rgb]{0.31,0.60,0.02}{#1}}
\newcommand{\CommentTok}[1]{\textcolor[rgb]{0.56,0.35,0.01}{\textit{#1}}}
\newcommand{\CommentVarTok}[1]{\textcolor[rgb]{0.56,0.35,0.01}{\textbf{\textit{#1}}}}
\newcommand{\ConstantTok}[1]{\textcolor[rgb]{0.00,0.00,0.00}{#1}}
\newcommand{\ControlFlowTok}[1]{\textcolor[rgb]{0.13,0.29,0.53}{\textbf{#1}}}
\newcommand{\DataTypeTok}[1]{\textcolor[rgb]{0.13,0.29,0.53}{#1}}
\newcommand{\DecValTok}[1]{\textcolor[rgb]{0.00,0.00,0.81}{#1}}
\newcommand{\DocumentationTok}[1]{\textcolor[rgb]{0.56,0.35,0.01}{\textbf{\textit{#1}}}}
\newcommand{\ErrorTok}[1]{\textcolor[rgb]{0.64,0.00,0.00}{\textbf{#1}}}
\newcommand{\ExtensionTok}[1]{#1}
\newcommand{\FloatTok}[1]{\textcolor[rgb]{0.00,0.00,0.81}{#1}}
\newcommand{\FunctionTok}[1]{\textcolor[rgb]{0.00,0.00,0.00}{#1}}
\newcommand{\ImportTok}[1]{#1}
\newcommand{\InformationTok}[1]{\textcolor[rgb]{0.56,0.35,0.01}{\textbf{\textit{#1}}}}
\newcommand{\KeywordTok}[1]{\textcolor[rgb]{0.13,0.29,0.53}{\textbf{#1}}}
\newcommand{\NormalTok}[1]{#1}
\newcommand{\OperatorTok}[1]{\textcolor[rgb]{0.81,0.36,0.00}{\textbf{#1}}}
\newcommand{\OtherTok}[1]{\textcolor[rgb]{0.56,0.35,0.01}{#1}}
\newcommand{\PreprocessorTok}[1]{\textcolor[rgb]{0.56,0.35,0.01}{\textit{#1}}}
\newcommand{\RegionMarkerTok}[1]{#1}
\newcommand{\SpecialCharTok}[1]{\textcolor[rgb]{0.00,0.00,0.00}{#1}}
\newcommand{\SpecialStringTok}[1]{\textcolor[rgb]{0.31,0.60,0.02}{#1}}
\newcommand{\StringTok}[1]{\textcolor[rgb]{0.31,0.60,0.02}{#1}}
\newcommand{\VariableTok}[1]{\textcolor[rgb]{0.00,0.00,0.00}{#1}}
\newcommand{\VerbatimStringTok}[1]{\textcolor[rgb]{0.31,0.60,0.02}{#1}}
\newcommand{\WarningTok}[1]{\textcolor[rgb]{0.56,0.35,0.01}{\textbf{\textit{#1}}}}
\usepackage{graphicx,grffile}
\makeatletter
\def\maxwidth{\ifdim\Gin@nat@width>\linewidth\linewidth\else\Gin@nat@width\fi}
\def\maxheight{\ifdim\Gin@nat@height>\textheight\textheight\else\Gin@nat@height\fi}
\makeatother
% Scale images if necessary, so that they will not overflow the page
% margins by default, and it is still possible to overwrite the defaults
% using explicit options in \includegraphics[width, height, ...]{}
\setkeys{Gin}{width=\maxwidth,height=\maxheight,keepaspectratio}
\IfFileExists{parskip.sty}{%
\usepackage{parskip}
}{% else
\setlength{\parindent}{0pt}
\setlength{\parskip}{6pt plus 2pt minus 1pt}
}
\setlength{\emergencystretch}{3em}  % prevent overfull lines
\providecommand{\tightlist}{%
  \setlength{\itemsep}{0pt}\setlength{\parskip}{0pt}}
\setcounter{secnumdepth}{0}
% Redefines (sub)paragraphs to behave more like sections
\ifx\paragraph\undefined\else
\let\oldparagraph\paragraph
\renewcommand{\paragraph}[1]{\oldparagraph{#1}\mbox{}}
\fi
\ifx\subparagraph\undefined\else
\let\oldsubparagraph\subparagraph
\renewcommand{\subparagraph}[1]{\oldsubparagraph{#1}\mbox{}}
\fi

%%% Use protect on footnotes to avoid problems with footnotes in titles
\let\rmarkdownfootnote\footnote%
\def\footnote{\protect\rmarkdownfootnote}

%%% Change title format to be more compact
\usepackage{titling}

% Create subtitle command for use in maketitle
\providecommand{\subtitle}[1]{
  \posttitle{
    \begin{center}\large#1\end{center}
    }
}

\setlength{\droptitle}{-2em}

  \title{Predictions using the Weight Lifting Exercises Dataset}
    \pretitle{\vspace{\droptitle}\centering\huge}
  \posttitle{\par}
    \author{Tom Lous}
    \preauthor{\centering\large\emph}
  \postauthor{\par}
      \predate{\centering\large\emph}
  \postdate{\par}
    \date{17 May 2016}


\begin{document}
\maketitle

\hypertarget{executive-summary}{%
\subsection{Executive Summary}\label{executive-summary}}

Based on a dataset provide by HAR
\url{http://groupware.les.inf.puc-rio.br/har} we will try to train a
predictive model to predict what exercise was performed using a dataset
with 159 features

We'll take the following steps:

\begin{itemize}
\tightlist
\item
  Process the data, for use of this project
\item
  Explore the data, especially focussing on the two paramaters we are
  interested in
\item
  Model selection, where we try different models to help us answer our
  questions
\item
  Model examination, to see wether our best model holds up to our
  standards
\item
  A Conclusion where we answer the questions based on the data
\item
  Predicting the classification of the model on test set
\end{itemize}

\hypertarget{processing}{%
\subsection{Processing}\label{processing}}

First change `am' to factor (0 = automatic, 1 = manual) And make
cylinders a factor as well (since it is not continious)

\begin{Shaded}
\begin{Highlighting}[]
\NormalTok{training.raw <-}\StringTok{ }\KeywordTok{read.csv}\NormalTok{(}\StringTok{"pml-training.csv"}\NormalTok{)}
\NormalTok{testing.raw <-}\StringTok{ }\KeywordTok{read.csv}\NormalTok{(}\StringTok{"pml-testing.csv"}\NormalTok{)}
\end{Highlighting}
\end{Shaded}

\hypertarget{exploratory-data-analyses}{%
\subsection{Exploratory data analyses}\label{exploratory-data-analyses}}

Look at the dimensions \& head of the dataset to get an idea

\begin{Shaded}
\begin{Highlighting}[]
\CommentTok{# Res 1}
\KeywordTok{dim}\NormalTok{(training.raw)}
\end{Highlighting}
\end{Shaded}

\begin{verbatim}
## [1] 19622   160
\end{verbatim}

\begin{Shaded}
\begin{Highlighting}[]
\CommentTok{# Res 2 - excluded because excessivness}
\CommentTok{# head(training.raw)}
\CommentTok{# Res 3 - excluded because excessivness}
\CommentTok{#str(training.raw)}
\CommentTok{# Res 4 - excluded because excessivness}
\CommentTok{#summary(training.raw)}
\end{Highlighting}
\end{Shaded}

What we see is a lot of data with NA / empty values. Let's remove those

\begin{Shaded}
\begin{Highlighting}[]
\NormalTok{maxNAPerc =}\StringTok{ }\DecValTok{20}
\NormalTok{maxNACount <-}\StringTok{ }\KeywordTok{nrow}\NormalTok{(training.raw) }\OperatorTok{/}\StringTok{ }\DecValTok{100} \OperatorTok{*}\StringTok{ }\NormalTok{maxNAPerc}
\NormalTok{removeColumns <-}\StringTok{ }\KeywordTok{which}\NormalTok{(}\KeywordTok{colSums}\NormalTok{(}\KeywordTok{is.na}\NormalTok{(training.raw) }\OperatorTok{|}\StringTok{ }\NormalTok{training.raw}\OperatorTok{==}\StringTok{""}\NormalTok{) }\OperatorTok{>}\StringTok{ }\NormalTok{maxNACount)}
\NormalTok{training.cleaned01 <-}\StringTok{ }\NormalTok{training.raw[,}\OperatorTok{-}\NormalTok{removeColumns]}
\NormalTok{testing.cleaned01 <-}\StringTok{ }\NormalTok{testing.raw[,}\OperatorTok{-}\NormalTok{removeColumns]}
\end{Highlighting}
\end{Shaded}

Also remove all time related data, since we won't use those

\begin{Shaded}
\begin{Highlighting}[]
\NormalTok{removeColumns <-}\StringTok{ }\KeywordTok{grep}\NormalTok{(}\StringTok{"timestamp"}\NormalTok{, }\KeywordTok{names}\NormalTok{(training.cleaned01))}
\NormalTok{training.cleaned02 <-}\StringTok{ }\NormalTok{training.cleaned01[,}\OperatorTok{-}\KeywordTok{c}\NormalTok{(}\DecValTok{1}\NormalTok{, removeColumns )]}
\NormalTok{testing.cleaned02 <-}\StringTok{ }\NormalTok{testing.cleaned01[,}\OperatorTok{-}\KeywordTok{c}\NormalTok{(}\DecValTok{1}\NormalTok{, removeColumns )]}
\end{Highlighting}
\end{Shaded}

Then convert all factors to integers

\begin{Shaded}
\begin{Highlighting}[]
\NormalTok{classeLevels <-}\StringTok{ }\KeywordTok{levels}\NormalTok{(training.cleaned02}\OperatorTok{$}\NormalTok{classe)}
\NormalTok{training.cleaned03 <-}\StringTok{ }\KeywordTok{data.frame}\NormalTok{(}\KeywordTok{data.matrix}\NormalTok{(training.cleaned02))}
\NormalTok{training.cleaned03}\OperatorTok{$}\NormalTok{classe <-}\StringTok{ }\KeywordTok{factor}\NormalTok{(training.cleaned03}\OperatorTok{$}\NormalTok{classe, }\DataTypeTok{labels=}\NormalTok{classeLevels)}
\NormalTok{testing.cleaned03 <-}\StringTok{ }\KeywordTok{data.frame}\NormalTok{(}\KeywordTok{data.matrix}\NormalTok{(testing.cleaned02))}
\end{Highlighting}
\end{Shaded}

Finally set the dataset to be explored

\begin{Shaded}
\begin{Highlighting}[]
\NormalTok{training.cleaned <-}\StringTok{ }\NormalTok{training.cleaned03}
\NormalTok{testing.cleaned <-}\StringTok{ }\NormalTok{testing.cleaned03}
\end{Highlighting}
\end{Shaded}

\hypertarget{exploratory-data-analyses-1}{%
\subsection{Exploratory data
analyses}\label{exploratory-data-analyses-1}}

Since the test set provided is the the ultimate validation set, we will
split the current training in a test and train set to work with.

\begin{Shaded}
\begin{Highlighting}[]
\KeywordTok{set.seed}\NormalTok{(}\DecValTok{19791108}\NormalTok{)}
\KeywordTok{library}\NormalTok{(caret)}
\NormalTok{classeIndex <-}\StringTok{ }\KeywordTok{which}\NormalTok{(}\KeywordTok{names}\NormalTok{(training.cleaned) }\OperatorTok{==}\StringTok{ "classe"}\NormalTok{)}
\NormalTok{partition <-}\StringTok{ }\KeywordTok{createDataPartition}\NormalTok{(}\DataTypeTok{y=}\NormalTok{training.cleaned}\OperatorTok{$}\NormalTok{classe, }\DataTypeTok{p=}\FloatTok{0.75}\NormalTok{, }\DataTypeTok{list=}\OtherTok{FALSE}\NormalTok{)}
\NormalTok{training.subSetTrain <-}\StringTok{ }\NormalTok{training.cleaned[partition, ]}
\NormalTok{training.subSetTest <-}\StringTok{ }\NormalTok{training.cleaned[}\OperatorTok{-}\NormalTok{partition, ]}
\end{Highlighting}
\end{Shaded}

What are some fields that have high correlations with the classe?

\begin{Shaded}
\begin{Highlighting}[]
\NormalTok{correlations <-}\StringTok{ }\KeywordTok{cor}\NormalTok{(training.subSetTrain[, }\OperatorTok{-}\NormalTok{classeIndex], }\KeywordTok{as.numeric}\NormalTok{(training.subSetTrain}\OperatorTok{$}\NormalTok{classe))}
\NormalTok{bestCorrelations <-}\StringTok{ }\KeywordTok{subset}\NormalTok{(}\KeywordTok{as.data.frame}\NormalTok{(}\KeywordTok{as.table}\NormalTok{(correlations)), }\KeywordTok{abs}\NormalTok{(Freq)}\OperatorTok{>}\FloatTok{0.3}\NormalTok{)}
\NormalTok{bestCorrelations}
\end{Highlighting}
\end{Shaded}

\begin{verbatim}
##             Var1 Var2     Freq
## 44 pitch_forearm    A 0.336018
\end{verbatim}

Even the best correlations with classe are hardly above 0.3 Let's check
visually if there is indeed hard to use these 2 as possible simple
linear predictors.

\begin{Shaded}
\begin{Highlighting}[]
\KeywordTok{library}\NormalTok{(Rmisc)}
\KeywordTok{library}\NormalTok{(ggplot2)}
\NormalTok{p1 <-}\StringTok{ }\KeywordTok{ggplot}\NormalTok{(training.subSetTrain, }\KeywordTok{aes}\NormalTok{(classe,pitch_forearm)) }\OperatorTok{+}\StringTok{ }
\StringTok{  }\KeywordTok{geom_boxplot}\NormalTok{(}\KeywordTok{aes}\NormalTok{(}\DataTypeTok{fill=}\NormalTok{classe))}
\NormalTok{p2 <-}\StringTok{ }\KeywordTok{ggplot}\NormalTok{(training.subSetTrain, }\KeywordTok{aes}\NormalTok{(classe, magnet_arm_x)) }\OperatorTok{+}\StringTok{ }
\StringTok{  }\KeywordTok{geom_boxplot}\NormalTok{(}\KeywordTok{aes}\NormalTok{(}\DataTypeTok{fill=}\NormalTok{classe))}
\KeywordTok{multiplot}\NormalTok{(p1,p2,}\DataTypeTok{cols=}\DecValTok{2}\NormalTok{)}
\end{Highlighting}
\end{Shaded}

\includegraphics{Final_Project_files/figure-latex/unnamed-chunk-9-1.pdf}

Clearly there is no hard seperation of classes possible using only these
`highly' correlated features. Let's train some models to get closer to a
way of predicting these classe's

\hypertarget{model-selection}{%
\subsection{Model selection}\label{model-selection}}

Let's identify variables with high correlations amongst each other in
our set, so we can possibly exclude them from the pca or training.

We will check afterwards if these modifications to the dataset make the
model more accurate (and perhaps even faster)

\begin{Shaded}
\begin{Highlighting}[]
\KeywordTok{library}\NormalTok{(corrplot)}
\NormalTok{correlationMatrix <-}\StringTok{ }\KeywordTok{cor}\NormalTok{(training.subSetTrain[, }\OperatorTok{-}\NormalTok{classeIndex])}
\NormalTok{highlyCorrelated <-}\StringTok{ }\KeywordTok{findCorrelation}\NormalTok{(correlationMatrix, }\DataTypeTok{cutoff=}\FloatTok{0.9}\NormalTok{, }\DataTypeTok{exact=}\OtherTok{TRUE}\NormalTok{)}
\NormalTok{excludeColumns <-}\StringTok{ }\KeywordTok{c}\NormalTok{(highlyCorrelated, classeIndex)}
\KeywordTok{corrplot}\NormalTok{(correlationMatrix, }\DataTypeTok{method=}\StringTok{"color"}\NormalTok{, }\DataTypeTok{type=}\StringTok{"lower"}\NormalTok{, }\DataTypeTok{order=}\StringTok{"hclust"}\NormalTok{, }\DataTypeTok{tl.cex=}\FloatTok{0.70}\NormalTok{, }\DataTypeTok{tl.col=}\StringTok{"black"}\NormalTok{, }\DataTypeTok{tl.srt =} \DecValTok{45}\NormalTok{, }\DataTypeTok{diag =} \OtherTok{FALSE}\NormalTok{)}
\end{Highlighting}
\end{Shaded}

\includegraphics{Final_Project_files/figure-latex/unnamed-chunk-10-1.pdf}

We see that there are some features that aree quite correlated with each
other. We will have a model with these excluded. Also we'll try and
reduce the features by running PCA on all and the excluded subset of the
features

\begin{Shaded}
\begin{Highlighting}[]
\NormalTok{pcaPreProcess.all <-}\StringTok{ }\KeywordTok{preProcess}\NormalTok{(training.subSetTrain[, }\OperatorTok{-}\NormalTok{classeIndex], }\DataTypeTok{method =} \StringTok{"pca"}\NormalTok{, }\DataTypeTok{thresh =} \FloatTok{0.99}\NormalTok{)}
\NormalTok{training.subSetTrain.pca.all <-}\StringTok{ }\KeywordTok{predict}\NormalTok{(pcaPreProcess.all, training.subSetTrain[, }\OperatorTok{-}\NormalTok{classeIndex])}
\NormalTok{training.subSetTest.pca.all <-}\StringTok{ }\KeywordTok{predict}\NormalTok{(pcaPreProcess.all, training.subSetTest[, }\OperatorTok{-}\NormalTok{classeIndex])}
\NormalTok{testing.pca.all <-}\StringTok{ }\KeywordTok{predict}\NormalTok{(pcaPreProcess.all, testing.cleaned[, }\OperatorTok{-}\NormalTok{classeIndex])}
\NormalTok{pcaPreProcess.subset <-}\StringTok{ }\KeywordTok{preProcess}\NormalTok{(training.subSetTrain[, }\OperatorTok{-}\NormalTok{excludeColumns], }\DataTypeTok{method =} \StringTok{"pca"}\NormalTok{, }\DataTypeTok{thresh =} \FloatTok{0.99}\NormalTok{)}
\NormalTok{training.subSetTrain.pca.subset <-}\StringTok{ }\KeywordTok{predict}\NormalTok{(pcaPreProcess.subset, training.subSetTrain[, }\OperatorTok{-}\NormalTok{excludeColumns])}
\NormalTok{training.subSetTest.pca.subset <-}\StringTok{ }\KeywordTok{predict}\NormalTok{(pcaPreProcess.subset, training.subSetTest[, }\OperatorTok{-}\NormalTok{excludeColumns])}
\NormalTok{testing.pca.subset <-}\StringTok{ }\KeywordTok{predict}\NormalTok{(pcaPreProcess.subset, testing.cleaned[, }\OperatorTok{-}\NormalTok{classeIndex])}
\end{Highlighting}
\end{Shaded}

Now we'll do some actual Random Forest training. We'll use 200 trees,
because I've already seen that the error rate doesn't decline a lot
after say 50 trees, but we still want to be thorough. Also we will time
each of the 4 random forest models to see if when all else is equal one
pops out as the faster one.

\begin{Shaded}
\begin{Highlighting}[]
\KeywordTok{library}\NormalTok{(randomForest)}
\NormalTok{ntree <-}\StringTok{ }\DecValTok{200} \CommentTok{#This is enough for great accuracy (trust me, I'm an engineer). }
\NormalTok{start <-}\StringTok{ }\KeywordTok{proc.time}\NormalTok{()}
\NormalTok{rfMod.cleaned <-}\StringTok{ }\KeywordTok{randomForest}\NormalTok{(}
  \DataTypeTok{x=}\NormalTok{training.subSetTrain[, }\OperatorTok{-}\NormalTok{classeIndex], }
  \DataTypeTok{y=}\NormalTok{training.subSetTrain}\OperatorTok{$}\NormalTok{classe,}
  \DataTypeTok{xtest=}\NormalTok{training.subSetTest[, }\OperatorTok{-}\NormalTok{classeIndex], }
  \DataTypeTok{ytest=}\NormalTok{training.subSetTest}\OperatorTok{$}\NormalTok{classe, }
  \DataTypeTok{ntree=}\NormalTok{ntree,}
  \DataTypeTok{keep.forest=}\OtherTok{TRUE}\NormalTok{,}
  \DataTypeTok{proximity=}\OtherTok{TRUE}\NormalTok{) }\CommentTok{#do.trace=TRUE}
\KeywordTok{proc.time}\NormalTok{() }\OperatorTok{-}\StringTok{ }\NormalTok{start}
\end{Highlighting}
\end{Shaded}

\begin{verbatim}
##    user  system elapsed 
##   75.62    0.64   76.36
\end{verbatim}

\begin{Shaded}
\begin{Highlighting}[]
\NormalTok{start <-}\StringTok{ }\KeywordTok{proc.time}\NormalTok{()}
\NormalTok{rfMod.exclude <-}\StringTok{ }\KeywordTok{randomForest}\NormalTok{(}
  \DataTypeTok{x=}\NormalTok{training.subSetTrain[, }\OperatorTok{-}\NormalTok{excludeColumns], }
  \DataTypeTok{y=}\NormalTok{training.subSetTrain}\OperatorTok{$}\NormalTok{classe,}
  \DataTypeTok{xtest=}\NormalTok{training.subSetTest[, }\OperatorTok{-}\NormalTok{excludeColumns], }
  \DataTypeTok{ytest=}\NormalTok{training.subSetTest}\OperatorTok{$}\NormalTok{classe, }
  \DataTypeTok{ntree=}\NormalTok{ntree,}
  \DataTypeTok{keep.forest=}\OtherTok{TRUE}\NormalTok{,}
  \DataTypeTok{proximity=}\OtherTok{TRUE}\NormalTok{) }\CommentTok{#do.trace=TRUE}
\KeywordTok{proc.time}\NormalTok{() }\OperatorTok{-}\StringTok{ }\NormalTok{start}
\end{Highlighting}
\end{Shaded}

\begin{verbatim}
##    user  system elapsed 
##   73.63    0.75   74.37
\end{verbatim}

\begin{Shaded}
\begin{Highlighting}[]
\NormalTok{start <-}\StringTok{ }\KeywordTok{proc.time}\NormalTok{()}
\NormalTok{rfMod.pca.all <-}\StringTok{ }\KeywordTok{randomForest}\NormalTok{(}
  \DataTypeTok{x=}\NormalTok{training.subSetTrain.pca.all, }
  \DataTypeTok{y=}\NormalTok{training.subSetTrain}\OperatorTok{$}\NormalTok{classe,}
  \DataTypeTok{xtest=}\NormalTok{training.subSetTest.pca.all, }
  \DataTypeTok{ytest=}\NormalTok{training.subSetTest}\OperatorTok{$}\NormalTok{classe, }
  \DataTypeTok{ntree=}\NormalTok{ntree,}
  \DataTypeTok{keep.forest=}\OtherTok{TRUE}\NormalTok{,}
  \DataTypeTok{proximity=}\OtherTok{TRUE}\NormalTok{) }\CommentTok{#do.trace=TRUE}
\KeywordTok{proc.time}\NormalTok{() }\OperatorTok{-}\StringTok{ }\NormalTok{start}
\end{Highlighting}
\end{Shaded}

\begin{verbatim}
##    user  system elapsed 
##   72.08    0.83   72.94
\end{verbatim}

\begin{Shaded}
\begin{Highlighting}[]
\NormalTok{start <-}\StringTok{ }\KeywordTok{proc.time}\NormalTok{()}
\NormalTok{rfMod.pca.subset <-}\StringTok{ }\KeywordTok{randomForest}\NormalTok{(}
  \DataTypeTok{x=}\NormalTok{training.subSetTrain.pca.subset, }
  \DataTypeTok{y=}\NormalTok{training.subSetTrain}\OperatorTok{$}\NormalTok{classe,}
  \DataTypeTok{xtest=}\NormalTok{training.subSetTest.pca.subset, }
  \DataTypeTok{ytest=}\NormalTok{training.subSetTest}\OperatorTok{$}\NormalTok{classe, }
  \DataTypeTok{ntree=}\NormalTok{ntree,}
  \DataTypeTok{keep.forest=}\OtherTok{TRUE}\NormalTok{,}
  \DataTypeTok{proximity=}\OtherTok{TRUE}\NormalTok{) }\CommentTok{#do.trace=TRUE}
\KeywordTok{proc.time}\NormalTok{() }\OperatorTok{-}\StringTok{ }\NormalTok{start}
\end{Highlighting}
\end{Shaded}

\begin{verbatim}
##    user  system elapsed 
##   71.64    2.78   74.81
\end{verbatim}

\hypertarget{model-examination}{%
\subsection{Model examination}\label{model-examination}}

Now that we have 4 trained models, we will check the accuracies of each.

\begin{Shaded}
\begin{Highlighting}[]
\NormalTok{rfMod.cleaned}
\end{Highlighting}
\end{Shaded}

\begin{verbatim}
## 
## Call:
##  randomForest(x = training.subSetTrain[, -classeIndex], y = training.subSetTrain$classe,      xtest = training.subSetTest[, -classeIndex], ytest = training.subSetTest$classe,      ntree = ntree, proximity = TRUE, keep.forest = TRUE) 
##                Type of random forest: classification
##                      Number of trees: 200
## No. of variables tried at each split: 7
## 
##         OOB estimate of  error rate: 0.28%
## Confusion matrix:
##      A    B    C    D    E  class.error
## A 4184    0    0    0    1 0.0002389486
## B    5 2841    2    0    0 0.0024578652
## C    0   10 2557    0    0 0.0038955980
## D    0    0   17 2395    0 0.0070480929
## E    0    0    0    6 2700 0.0022172949
##                 Test set error rate: 0.31%
## Confusion matrix:
##      A   B   C   D   E class.error
## A 1395   0   0   0   0 0.000000000
## B    4 942   3   0   0 0.007376185
## C    0   2 853   0   0 0.002339181
## D    0   0   1 802   1 0.002487562
## E    0   0   0   4 897 0.004439512
\end{verbatim}

\begin{Shaded}
\begin{Highlighting}[]
\NormalTok{rfMod.cleaned.training.acc <-}\StringTok{ }\KeywordTok{round}\NormalTok{(}\DecValTok{1}\OperatorTok{-}\KeywordTok{sum}\NormalTok{(rfMod.cleaned}\OperatorTok{$}\NormalTok{confusion[, }\StringTok{'class.error'}\NormalTok{]),}\DecValTok{3}\NormalTok{)}
\KeywordTok{paste0}\NormalTok{(}\StringTok{"Accuracy on training: "}\NormalTok{,rfMod.cleaned.training.acc)}
\end{Highlighting}
\end{Shaded}

\begin{verbatim}
## [1] "Accuracy on training: 0.984"
\end{verbatim}

\begin{Shaded}
\begin{Highlighting}[]
\NormalTok{rfMod.cleaned.testing.acc <-}\StringTok{ }\KeywordTok{round}\NormalTok{(}\DecValTok{1}\OperatorTok{-}\KeywordTok{sum}\NormalTok{(rfMod.cleaned}\OperatorTok{$}\NormalTok{test}\OperatorTok{$}\NormalTok{confusion[, }\StringTok{'class.error'}\NormalTok{]),}\DecValTok{3}\NormalTok{)}
\KeywordTok{paste0}\NormalTok{(}\StringTok{"Accuracy on testing: "}\NormalTok{,rfMod.cleaned.testing.acc)}
\end{Highlighting}
\end{Shaded}

\begin{verbatim}
## [1] "Accuracy on testing: 0.983"
\end{verbatim}

\begin{Shaded}
\begin{Highlighting}[]
\NormalTok{rfMod.exclude}
\end{Highlighting}
\end{Shaded}

\begin{verbatim}
## 
## Call:
##  randomForest(x = training.subSetTrain[, -excludeColumns], y = training.subSetTrain$classe,      xtest = training.subSetTest[, -excludeColumns], ytest = training.subSetTest$classe,      ntree = ntree, proximity = TRUE, keep.forest = TRUE) 
##                Type of random forest: classification
##                      Number of trees: 200
## No. of variables tried at each split: 6
## 
##         OOB estimate of  error rate: 0.28%
## Confusion matrix:
##      A    B    C    D    E  class.error
## A 4184    1    0    0    0 0.0002389486
## B    4 2842    2    0    0 0.0021067416
## C    0   12 2555    0    0 0.0046747176
## D    0    0   15 2396    1 0.0066334992
## E    0    0    0    6 2700 0.0022172949
##                 Test set error rate: 0.29%
## Confusion matrix:
##      A   B   C   D   E class.error
## A 1395   0   0   0   0 0.000000000
## B    2 945   2   0   0 0.004214963
## C    0   4 851   0   0 0.004678363
## D    0   0   3 800   1 0.004975124
## E    0   0   0   2 899 0.002219756
\end{verbatim}

\begin{Shaded}
\begin{Highlighting}[]
\NormalTok{rfMod.exclude.training.acc <-}\StringTok{ }\KeywordTok{round}\NormalTok{(}\DecValTok{1}\OperatorTok{-}\KeywordTok{sum}\NormalTok{(rfMod.exclude}\OperatorTok{$}\NormalTok{confusion[, }\StringTok{'class.error'}\NormalTok{]),}\DecValTok{3}\NormalTok{)}
\KeywordTok{paste0}\NormalTok{(}\StringTok{"Accuracy on training: "}\NormalTok{,rfMod.exclude.training.acc)}
\end{Highlighting}
\end{Shaded}

\begin{verbatim}
## [1] "Accuracy on training: 0.984"
\end{verbatim}

\begin{Shaded}
\begin{Highlighting}[]
\NormalTok{rfMod.exclude.testing.acc <-}\StringTok{ }\KeywordTok{round}\NormalTok{(}\DecValTok{1}\OperatorTok{-}\KeywordTok{sum}\NormalTok{(rfMod.exclude}\OperatorTok{$}\NormalTok{test}\OperatorTok{$}\NormalTok{confusion[, }\StringTok{'class.error'}\NormalTok{]),}\DecValTok{3}\NormalTok{)}
\KeywordTok{paste0}\NormalTok{(}\StringTok{"Accuracy on testing: "}\NormalTok{,rfMod.exclude.testing.acc)}
\end{Highlighting}
\end{Shaded}

\begin{verbatim}
## [1] "Accuracy on testing: 0.984"
\end{verbatim}

\begin{Shaded}
\begin{Highlighting}[]
\NormalTok{rfMod.pca.all}
\end{Highlighting}
\end{Shaded}

\begin{verbatim}
## 
## Call:
##  randomForest(x = training.subSetTrain.pca.all, y = training.subSetTrain$classe,      xtest = training.subSetTest.pca.all, ytest = training.subSetTest$classe,      ntree = ntree, proximity = TRUE, keep.forest = TRUE) 
##                Type of random forest: classification
##                      Number of trees: 200
## No. of variables tried at each split: 6
## 
##         OOB estimate of  error rate: 2.08%
## Confusion matrix:
##      A    B    C    D    E class.error
## A 4162   11    1    8    3 0.005495818
## B   49 2775   18    2    4 0.025632022
## C    5   30 2509   21    2 0.022594468
## D    3    1  100 2300    8 0.046434494
## E    1    7   14   18 2666 0.014781966
##                 Test set error rate: 1.71%
## Confusion matrix:
##      A   B   C   D   E class.error
## A 1387   4   0   3   1 0.005734767
## B   14 930   5   0   0 0.020021075
## C    1  11 836   6   1 0.022222222
## D    1   0  23 777   3 0.033582090
## E    0   0   7   4 890 0.012208657
\end{verbatim}

\begin{Shaded}
\begin{Highlighting}[]
\NormalTok{rfMod.pca.all.training.acc <-}\StringTok{ }\KeywordTok{round}\NormalTok{(}\DecValTok{1}\OperatorTok{-}\KeywordTok{sum}\NormalTok{(rfMod.pca.all}\OperatorTok{$}\NormalTok{confusion[, }\StringTok{'class.error'}\NormalTok{]),}\DecValTok{3}\NormalTok{)}
\KeywordTok{paste0}\NormalTok{(}\StringTok{"Accuracy on training: "}\NormalTok{,rfMod.pca.all.training.acc)}
\end{Highlighting}
\end{Shaded}

\begin{verbatim}
## [1] "Accuracy on training: 0.885"
\end{verbatim}

\begin{Shaded}
\begin{Highlighting}[]
\NormalTok{rfMod.pca.all.testing.acc <-}\StringTok{ }\KeywordTok{round}\NormalTok{(}\DecValTok{1}\OperatorTok{-}\KeywordTok{sum}\NormalTok{(rfMod.pca.all}\OperatorTok{$}\NormalTok{test}\OperatorTok{$}\NormalTok{confusion[, }\StringTok{'class.error'}\NormalTok{]),}\DecValTok{3}\NormalTok{)}
\KeywordTok{paste0}\NormalTok{(}\StringTok{"Accuracy on testing: "}\NormalTok{,rfMod.pca.all.testing.acc)}
\end{Highlighting}
\end{Shaded}

\begin{verbatim}
## [1] "Accuracy on testing: 0.906"
\end{verbatim}

\begin{Shaded}
\begin{Highlighting}[]
\NormalTok{rfMod.pca.subset}
\end{Highlighting}
\end{Shaded}

\begin{verbatim}
## 
## Call:
##  randomForest(x = training.subSetTrain.pca.subset, y = training.subSetTrain$classe,      xtest = training.subSetTest.pca.subset, ytest = training.subSetTest$classe,      ntree = ntree, proximity = TRUE, keep.forest = TRUE) 
##                Type of random forest: classification
##                      Number of trees: 200
## No. of variables tried at each split: 6
## 
##         OOB estimate of  error rate: 2.43%
## Confusion matrix:
##      A    B    C    D    E class.error
## A 4155    8    9   10    3 0.007168459
## B   69 2741   29    5    4 0.037570225
## C    2   33 2510   20    2 0.022204908
## D    7    3  104 2292    6 0.049751244
## E    2    9   20   13 2662 0.016260163
##                 Test set error rate: 1.96%
## Confusion matrix:
##      A   B   C   D   E class.error
## A 1387   3   2   3   0 0.005734767
## B   18 924   7   0   0 0.026343519
## C    0  12 838   4   1 0.019883041
## D    0   1  27 774   2 0.037313433
## E    1   3   9   3 885 0.017758047
\end{verbatim}

\begin{Shaded}
\begin{Highlighting}[]
\NormalTok{rfMod.pca.subset.training.acc <-}\StringTok{ }\KeywordTok{round}\NormalTok{(}\DecValTok{1}\OperatorTok{-}\KeywordTok{sum}\NormalTok{(rfMod.pca.subset}\OperatorTok{$}\NormalTok{confusion[, }\StringTok{'class.error'}\NormalTok{]),}\DecValTok{3}\NormalTok{)}
\KeywordTok{paste0}\NormalTok{(}\StringTok{"Accuracy on training: "}\NormalTok{,rfMod.pca.subset.training.acc)}
\end{Highlighting}
\end{Shaded}

\begin{verbatim}
## [1] "Accuracy on training: 0.867"
\end{verbatim}

\begin{Shaded}
\begin{Highlighting}[]
\NormalTok{rfMod.pca.subset.testing.acc <-}\StringTok{ }\KeywordTok{round}\NormalTok{(}\DecValTok{1}\OperatorTok{-}\KeywordTok{sum}\NormalTok{(rfMod.pca.subset}\OperatorTok{$}\NormalTok{test}\OperatorTok{$}\NormalTok{confusion[, }\StringTok{'class.error'}\NormalTok{]),}\DecValTok{3}\NormalTok{)}
\KeywordTok{paste0}\NormalTok{(}\StringTok{"Accuracy on testing: "}\NormalTok{,rfMod.pca.subset.testing.acc)}
\end{Highlighting}
\end{Shaded}

\begin{verbatim}
## [1] "Accuracy on testing: 0.893"
\end{verbatim}

\hypertarget{conclusion}{%
\subsection{Conclusion}\label{conclusion}}

This concludes that nor PCA doesn't have a positive of the accuracy (or
the process time for that matter) The \texttt{rfMod.exclude} perform's
slightly better then the `rfMod.cleaned'

We'll stick with the \texttt{rfMod.exclude} model as the best model to
use for predicting the test set. Because with an accuracy of 98.7\% and
an estimated OOB error rate of 0.23\% this is the best model.


\end{document}
